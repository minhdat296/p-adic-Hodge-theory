\chapter{Introduction}
    The subject that is nowadays known as \say{Hodge theory} is - as least as far as its motivic aspect is concerned - essentially the study of cohomology theories (\'etale, de Rham, crystalline, etc.) as topological invariants of (algebraic and analytic) varieties along with the various relationships between said cohomology theories. As it stands currently, Hodge theory consists of two parallel subfields, namely the theories over the complex numbers $\bbC$ and that over non-archimedean fields such as $\Q_p$, bot of which could be seen as probational steps towards a more grander all-unifying incarnation of Hodge theory over number fields, which for various technical reasons might be the key technical toolbox for tackling some of the deepest networks of conjectures in modern algebraic number theory, like the Global Langlands Correspondence over $\Q$. For algebraic varieties over the complex numbers $\bbC$ (or more precisely, their associated complex manifolds), the story is relatively simple: due to there being only two interesting Grothendieck topologies, namely the \'etale topology on smooth varieties and the complex-analytic topology on their corresponding complex manifolds, there exists only one fundamental comparison theorem, that being the Riemann-Hilbert Correspondence, which more-or-less relates \'etale cohomology of local systems and de Rham cohomology of their associated complex manifolds. Over non-archimedean local fields such as $\Q_p$ or $\F_p(\!(t)\!)$, however, the story is much more complicated; here, there is a plethora of cohomology theories, and since we would like for there to exist pairwise comparison theorems amongst them, we will have to look beyond attempting to simply compare the various Grothendieck topologies. Moreoever, due to $p$-adic Hodge theory being composed of two different facades, namely the theories over mixed characteristic fields such as $\Q_p$ and equicharacteristic fields such as $\F_p(\!(t)\!)$, we will also have to worry about how certain cohomology theories behave poorly over positive characteristics (e.g. de Rham cohomology) and as such require replacements (e.g. crystalline cohomology), which might be full of their own stock of shortcomings and technical difficulties. Luckily, geometry over non-archimedean fields (and particularly those of positive characteristics) is a much richer theory than its complex counterpart and as such, there are many intermediary objects that one might try to make use of in order to \say{interpolate} the various cohomoloy theories and thereby comparing them indirectly; this, incidentally, is an approach to $p$-adic Hodge theory that has now gained mainstream popularity.
    
    \section{The abelian theory and cohomology theories for \texorpdfstring{$p$}{}-adic varieties}
    
    \section{The non-abelian theory, almost purity, and monodromy representations}